\documentclass[a4paper, 12pt]{article}
\usepackage[utf8]{inputenc}
\usepackage[russian]{babel}
\usepackage{setspace,amsmath}
\usepackage[left=20mm, top=15mm, right=15mm, bottom=15mm, nohead, footskip=10mm]{geometry} % настройки полей документа
\usepackage{tocloft}
\usepackage{hyperref}
\hypersetup{
    colorlinks,
    citecolor=black,
    filecolor=black,
    linkcolor=black,
    urlcolor=black
}
\renewcommand{\cftpartleader}{\cftdotfill{\cftdotsep}} % for parts
% \renewcommand{\cftchapleader}{\cftdotfill{\cftdotsep}} % for chapters
\renewcommand{\cftsecleader}{\cftdotfill{\cftdotsep}}
\begin{document} % начало документа
\begingroup
    \fontsize{14pt}{14pt}\selectfont
% НАЧАЛО ТИТУЛЬНОГО ЛИСТА
\begin{center}
\hfill \break
\footnotesize{НАЦИОНАЛЬНЫЙ ИССЛЕДОВАТЕЛЬСКИЙ УНИВЕРСИТЕТ}\\
\small{\textbf{«ВЫСШАЯ ШКОЛА ЭКОНОМИКИ»}}\\
\hfill \break
\normalsize{Факультет Компьютерных Наук}\\
 \hfill \break
\normalsize{Департамент Программной Инженерии}\\
\hfill\break
\hfill \break
\hfill \break
\hfill \break
\hfill \break
\hfill \break
\hfill \break
\hfill \break
\normalsize{Контрольное Домашнее Задание\\
\hfill \break
по дисциплине «Алгоритмы и Структуры Данных»\\
\hfill \break
\textbf{ОТЧЕТ}}\\
\hfill \break
\hfill \break
\end{center}

\hfill \break
\hfill \break
\hfill \break
\hfill \break
\begin{flushright}
  \normalsize{Выполнил:}\\
  \normalsize{Студент 2 курса группы БПИ151}\\
  \normalsize{Куприрянов Кирилл Игорвеич}
\end{flushright}
\hfill \break
\hfill \break
\hfill \break
\hfill \break
\hfill \break
\hfill \break
\hfill \break
\begin{center} Москва 2016 \end{center}
\thispagestyle{empty} % выключаем отображение номера для этой страницы

% КОНЕЦ ТИТУЛЬНОГО ЛИСТА
\endgroup

\newpage
    \tableofcontents % Вывод содержания
\newpage

\newpage
\section{Постановка задачи}
Необходимо было реализовать с использованием языка $C++$ программы для архивирования и
разархивирования текстовых файлов. При этом использовать два известных алгоритма кодирования информации:

\begin{enumerate}
  \item Хаффмана (не адаптивный, простой)
  \item Шеннона-Фано
\end{enumerate}
Обе реализации поместить в одном файле main.cpp, содержащем соответствующие методы:
\begin{enumerate}
  \item метод архивирования, использующий алгоритм Хаффмана,
  вход: текстовый файл <name>.txt (кодировка UTF-8)
  выход: архивированный файл <name>.haff
  \item метод разархивирования, использующий алгоритм Хаффмана,
  вход: архивированный файл <name>.haff
  выход: разархивированный файл <name>-unz-h.txt (кодировка UTF-8)
  \item метод архивирования, использующий алгоритм Шеннона-Фано,
  вход: текстовый файл <name>.txt (кодировка UTF-8)
  выход: архивированный файл <name>.shan
  \item метод разархивирования, использующий алгоритм Шеннона-Фано.
  вход: архивированный файл <name>.shan
  выход: разархивированный файл <name>-unz-s.txt (кодировка UTF-8)
\end{enumerate}

Выбор алгоритма осуществляется с помощью флага командной строки.

Оба алгоритма работают в два прохода. Сначала строится таблица частот
встречаемости символов в конкретном архивируемом файле (кодируем только те
символы из набора допустимых, которые реально встречаются в файле). Затем строится кодовое дерево (не обязательное).
По нему (или по таблице кодов) и архивируется файл. Для
разархивирования алгоритмам потребуется знать таблицу, которая использовалась
при архивировании. Соответствующая таблица должна сохраняться в архивном файле
в самом его начале и использоваться при разархивировании. В начале пишется
количество различных символов n, имеющихся в кодируемом файле, а затем n пар (код
символа UTF-8, битовый код в архиве). Порядок — по убыванию частоты встречаемости
символа в кодируемом файле.

Провести вычислительный эксперимент с целью оценки реализованных
алгоритмов архивации / разархивации. Оценить количество элементарных операций
каждого алгоритма.
Для этого
\begin{enumerate}
  \item Подготовить тестовый набор из нескольких текстовых файлов разного объема
  (20, 40, 60, 80, 100 Кб; 1, 2, 3 Мб — всего 8 файлов) на разных языках (ru, en -
  кодировка UTF-8) с разным набором символов в каждом файле, а именно:
  \begin{enumerate}
    \item первый набор: символы латинского алфавита и пробел
    \item второй набор: символы из первого набора + символы русского алфавита
    \item третий набор: символы из второго набора + следующие знаки и
    спецсимволы: знаки арифметики „+ - * / =“, знаки препинания „. , ; : ? !“,
    „\% @ \# \$ & ~‘’, скобки разных типов „( ) [ ] { } < >“ , кавычки „„““),
  \end{enumerate}
  \item Измерить (экспериментально) количество операций (в рамках модели RAM (взять из
    лекционного материала)), выполняемых за время работы (архивирования,
    разархивирования) каждого алгоритма на нескольких различных (не менее
    трех) файлах для каждого размера входного файла и набора символов (итого
    получается $8\times3\times3 = 72$ эксперимента по архивированию и 72 по
    разархивированию для каждого алгоритма, т.е. Всего минимум $144\times2 = 288$).
    Для повышения достоверности результатов каждый эксперимент можно повторить несколько (5-10) раз на
    различных файлах (с одним возможным набором символов) одного размера с
    последующим усреднением результата.
\end{enumerate}

Подготовить отчет по итогам работы, содержащий постановку задачи, описание
алгоритмов и задействованных структур данных, описание реализации, обобщенные
результаты измерения эффективности алгоритмов, описание использованных
инструментов (например, если использовались скрипты автоматизации), выводы о
соответствии результатов экспериментальной проверки с теоретическими оценками
эффективности исследуемых алгоритмов.
Отчет также должен содержать измерения качества архивации (степень сжатия =
отношение размеров выходного и входного файлов), оценку связи между степенью
сжатия для различных входных файлов (как влияют объем, язык, набор символов, их
разнообразие?) и временем работы (количеством операций) для каждого алгоритма.



\end{document}
